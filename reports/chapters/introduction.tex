\chapter{Introduction}
\label{ch:into} % This how you label a chapter and the key (e.g., ch:into) will be used to refer this chapter ``Introduction'' later in the report. 
% the key ``ch:into'' can be used with command \ref{ch:intor} to refere this Chapter.
Today, heart problems are a major health concern affecting individuals worldwide. Many people 
are suffering from heart issues like heart disease, heart failure, and irregular heartbeats \cite{webb2015care}. Heart problems can affect people, not just physically but also emotionally. Those 
with heart conditions often find it difficult to live normally and face many difficulties.
Additionally, the financial side of managing heart problems adds an extra layer of challenges. 
Spotting heart-related issues early is crucial. It helps healthcare professionals to step in 
quickly, enhance patient outcomes, and ease the strain on healthcare resources. Early detection  
allows for timely intervention, potentially preventing the progression of heart conditions.To 
address the need for early detection, our project focuses on developing a machine learning 
model capable of accurately identifying the presence of heart diseases. In this endeavor, we 
utilize a heart-related issue dataset from \citep{janosi-1988}, sourced from the online 
repository UC Irvine. The data undergoes thorough cleaning and pre-processing to extract useful 
information essential for training the machine learning model. The machine learning algorithms 
employed, as highlighted by \citep{Sharma-2020}, include LR, NB, and 
RF classification. These algorithms have demonstrated effectiveness in detecting 
coronary artery disease by evaluating outputs based on various factors such as resting blood 
pressure, serum cholesterol, maximum heart rate achieved, and more. Furthermore, our project 
aims not only to detect heart-related issues but also to contribute valuable insights to the 
broader field of cardiovascular health. By leveraging advanced algorithms, we seek to ensure 
the effective prediction of heart-related problems, potentially revolutionizing the early 
diagnosis and management of cardiovascular conditions.
%%%%%%%%%%%%%%%%%%%%%%%%%%%%%%%%%%%%%%%%%%%%%%%%%%%%%%%%%%%%%%%%%%%%%%%%%%%%%%%%%%%
\section{Background}
\label{sec:into_back}
Our project focuses on addressing the issue of cardiovascular diseases in today's world, 
affecting everyone irrespective of their age. The primary motivation behind our work is to 
detect the heart-related diseases as early as possible. This identification helps doctors to 
start the treatment sooner, to improve patient results and effectively using the healthcare 
resources. For this, our project focuses on integrating technology into healthcare by using the 
abundant data in cardiology for predictive modeling. The primary goal is to develop an 
efficient heart disease prediction system by concentrating on predicting CAD with precision and effectiveness. So, we use different machine learning models such as 
LR, RF, NB. These models play a crucial role in 
predicting and understanding heart-related issues. The project commences with data preprocessing to extract essential patterns required for training the models, aligning with the objectives and research approach outlined in subsequent sections. Our 
project becomes significant as it can give better resources to doctors for finding and handling 
heart problems early on. We want to help make hearts healthier by explaining some crucial ideas 
and ways to use them in a simple way.

%%%%%%%%%%%%%%%%%%%%%%%%%%%%%%%%%%%%%%%%%%%%%%%%%%%%%%%%%%%%%%%%%%%%%%%%%%%%%%%%%%%
\section{Research Question}
\label{sec:intro_prob_art}
How can machine learning models, specifically Logistic Regression, Naïve Bayes, and Random 
Forest classification algorithms, be effectively utilized to develop a heart disease prediction 
system for early detection of Coronary Artery Disease, with a focus on improving patient 
outcomes and contributing to advancements in cardiovascular health?

%%%%%%%%%%%%%%%%%%%%%%%%%%%%%%%%%%%%%%%%%%%%%%%%%%%%%%%%%%%%%%%%%%%%%%%%%%%%%%%%%%%
\subsection{Aims and objectives}
\label{sec:intro_aims_obj}
 

\textbf{Aims:}To develop and implement an advanced heart disease prediction system, utilizing 
machine learning models for early detection of CAD, with the ultimate goal 
of enhancing patient outcomes and contributing to the ongoing global efforts in cardiovascular 
health.
\textbf{Objectives:}
\begin{itemize}
    \item Obtain and analyze the heart disease dataset, clean and preprocess the data for model training.
    \item Train LR classifier, optimizing meta-parameters for improved performance.
    \item Develop NB classifier, focusing on feature selection and parameter tuning.
    \item Utilize RF algorithm to construct decision tree ensembles, refining predictive capabilities.
    \item Integrate trained models into healthcare systems for real-time heart disease prediction.
    \item Provide healthcare professionals with valuable insights and resources for informed decision-making.
\end{itemize}



%%%%%%%%%%%%%%%%%%%%%%%%%%%%%%%%%%%%%%%%%%%%%%%%%%%%%%%%%%%%%%%%%%%%%%%%%%%%%%%%%%%
\section{Solution Approach}
\label{sec:intro_sol} % label of Org section
The solution approach involves a thorough step-by-step method designed to create an advanced system for predicting heart disease, specifically focusing on early detection of CAD.

\subsection{Dataset Acquisition and Preprocessing}
We begin by acquiring and analyzing the heart disease dataset, obtained from the research conducted by Janosi et al. [1], which is accessible through UC Irvine. The dataset is carefully cleaned and processed to identify and handle missing values, outliers, and inconsistencies.

\subsection{Model Training and Optimization}
\begin{itemize}
\item \textbf{Logistic Regression (LR):} We train the LR classifier, optimizing meta-parameters such as regularization strength and maximum iterations to improve performance. LR is chosen for its ability to effectively classify CAD based on key risk factors such as resting blood pressure, serum cholesterol, and maximum heart rate achieved.
\item \textbf{Naïve Bayes (NB):} The NB classifier is developed, focusing on feature selection and parameter tuning. NB is particularly suited for its simplicity and speed, making it an efficient model for heart disease prediction.
\item \textbf{Random Forest (RF):} We utilize the RF algorithm to construct decision tree ensembles, refining predictive capabilities. RF is selected for its ability to handle nonlinear relationships and interactions between features, improving the accuracy of CAD prediction.
\end{itemize}

\subsection{Model Evaluation and Integration}
The trained models are evaluated using specific metrics such as precision, recall, and the F1 score to assess their performance. Finally, the models are integrated into healthcare systems for real-time heart disease prediction, providing healthcare professionals with valuable insights and resources for informed decision-making.



%%%%%%%%%%%%%%%%%%%%%%%%%%%%%%%%%%%%%%%%%%%%%%%%%%%%%%%%%%%%%%%%%%%%%%%%%%%%%%%%%%%
\section{Summary of contributions and achievements} %  use this section 
\label{sec:intro_sum_results} % label of summary of results
In this project, we aimed to develop an advanced system for predicting heart disease using machine learning models, with the ultimate goal of enhancing patient outcomes and contributing to global efforts in cardiovascular health. We began by obtaining and analyzing a heart disease dataset, meticulously cleaning and preprocessing the data to prepare it for model training. We then trained three different machine learning models: Logistic Regression (LR), Naïve Bayes (NB), and Random Forest (RF). LR emerged as the top performer, achieving an accuracy of 87\%, indicating a high degree of correctness in its predictions. Our models not only accurately predict heart disease but also provide valuable insights into the risk factors associated with the condition, enabling healthcare professionals to make informed decisions and ultimately improving patient outcomes.

%%%%%%%%%%%%%%%%%%%%%%%%%%%%%%%%%%%%%%%%%%%%%%%%%%%%%%%%%%%%%%%%%%%%%%%%%%%%%%%%%%%

