\chapter{Conclusions and Future Work}
\label{ch:con}
\section{Conclusions}
In conclusion, this project aimed to develop machine learning models for predicting heart disease, with a primary focus on early detection of Coronary Artery Disease (CAD). Through the implementation of Logistic Regression, Random Forest, and Naive Bayes algorithms, we successfully trained models on a dataset containing relevant patient information. Our findings indicate that the Logistic Regression model achieved the highest accuracy and demonstrated a good balance between precision and recall. This suggests that machine learning algorithms, particularly Logistic Regression, hold promise as effective tools for predicting heart disease and contributing to improved patient outcomes. Overall, this project's central contributions lie in the successful implementation and evaluation of machine learning models for heart disease prediction, paving the way for future advancements in cardiovascular health.

\section{Future work}
While this project has made significant advancements in predicting heart disease, there are still opportunities for further exploration and improvement. Moving forward, it would be beneficial to conduct further research to refine the machine learning models and enhance their predictive capabilities. Additionally, exploring the integration of additional features or datasets could provide valuable insights into improving the accuracy and robustness of the models. Additionally, it would be helpful to study how well the prediction system works over a long time in real-life healthcare settings. In the future, we should keep improving and testing the prediction system to make sure it works well and can be trusted by doctors.