%Two resources useful for abstract writing.
% Guidance of how to write an abstract/summary provided by Nature: https://cbs.umn.edu/sites/cbs.umn.edu/files/public/downloads/Annotated_Nature_abstract.pdf %https://writingcenter.gmu.edu/guides/writing-an-abstract
\chapter*{\center \Large  Abstract}
%%%%%%%%%%%%%%%%%%%%%%%%%%%%%%%%%%%%%%
% Replace all text with your text
%%%%%%%%%%%%%%%%%%%%%%%%%%%%%%%%%%%
In Contemporary world Cardiovascular diseases have seen a rise, even affecting newborn. Detecting heart-related diseases prior is vital because it helps doctors start treatment sooner, leading to better results for patients and less strain on healthcare resources. With more and more people facing heart problems, it's crucial to have advanced predictive tools. Using the abundant data available in Cardiology, our project aims to integrate the technology into health care for predictive modelling. The primary goal of this project is to develop an efficient heart disease prediction system using various machine learning models to predict Coronary Artery Disease with utmost precision and effectiveness. We employed a dataset consisting of necessary patient information from online sources to train and validate our models. The first step is Cleaning and preprocessing data that allow us to find key patterns for training the models. The various machine learning models used are Logistic Regression, Random Forest, Naive Bayes. We evaluate these models using important measures like precision, which tells us how accurate positive predictions are; recall, which shows how well the models capture all actual positive cases; and the F1 score, which balances both precision and recall which can be assessed to provide the health sectors with a more efficient approach to detect heart related diseases.



%%%%%%%%%%%%%%%%%%%%%%%%%%%%%%%%%%%%%%%%%%%%%%%%%%%%%%%%%%%%%%%%%%%%%%%%%s
~\\[1cm]
\noindent % Provide your key words
\textbf{Keywords:} Logistic Regression, Random Forest, Naive Bayes , F1 score,Precision.


\vfill
\noindent
