\chapter{Literature Review}
\label{ch:lit_rev} %Label of the chapter lit rev. The key ``ch:lit_rev'' can be used with command \ref{ch:lit_rev} to refer this Chapter.

A literature review chapter can be organized in a few sections with appropriate titles. A literature review chapter might  contain the following:
\begin{enumerate}
    \item A review of the state-of-the-art (include theories and solutions) of the field of research.
    \item A description of the project in the context of existing literature and products/systems.
    \item An analysis of how the review is relevant to the intended application/system/problem.
    \item A critique of existing work compared with the intended work.
\end{enumerate}
Note that your literature review should demonstrate the significance of the project.

% PLEAE CHANGE THE TITLE of this section
\section{Example of in-text citation of references in \LaTeX} 
% Note the use of \cite{} and \citep{}
The references in a report relate your content with the relevant sources, papers, and the works of others. To include references in a report, we \textit{cite} them in the texts. In MS-Word, EndNote, or MS-Word references, or plain text as a list can be used. Similarly, in \LaTeX, you can use the ``thebibliography'' environment, which is similar to the plain text as a list arrangement like the MS word. However, In \LaTeX, the most convenient way is to use the BibTex, which takes the references in a particular format [see references.bib file of this template] and lists them in style [APA, Harvard, etc.] as we want with the help of proper packages.    

These are the examples of how to \textit{cite} external sources, seminal works, and research papers. In \LaTeX, if you use ``\textbf{BibTex}'' you do not have to worry much since the proper use of a bibliographystyle package like ``agsm for the Harvard style'' and little rectification of the content in a BiBText source file [In this template, BibTex are stored in the ``references.bib'' file], we can conveniently generate  a reference style. 

Take a note of the commands \textbackslash cite\{\} and \textbackslash citep\{\}. The command \textbackslash cite\{\} will write like ``Author et al. (2019)'' style for Harvard, APA and Chicago style. The command \textbackslash citep\{\} will write like ``(Author et al., 2019).'' Depending on how you construct a sentence, you need to use them smartly. Check the examples of \textbf{in-text citation} of sources listed here [This template recommends the \textbf{Harvard style} of referencing.]:
\begin{itemize}
    \item \cite{lamport1994latex} has written a comprehensive guide on writing in \LaTeX ~[Example of \textbackslash cite\{\} ].
    \item If \LaTeX~is used efficiently and effectively, it helps in writing a very high-quality project report~\citep{lamport1994latex} ~[Example of \textbackslash citep\{\} ].   
    \item A detailed APA, Harvard, and Chicago referencing style guide are available in~\citep{uor_refernce_style}.
\end{itemize}

\noindent 
Example of a numbered list:
\begin{enumerate}
    \item \cite{lamport1994latex} has written a comprehensive guide on writing in \LaTeX.
    \item If \LaTeX is used efficiently and effectively, it helps in writing a very high-quality project report~\citep{lamport1994latex}.   
\end{enumerate}

% PLEAE CHANGE THE TITLE of this section
\section{Example of ``risk'' of unintentional plagiarism}
Using other sources, ideas, and material always bring with it a risk of unintentional plagiarism. 

\noindent
\textbf{\color{red}MUST}: do read the university guidelines on the definition of plagiarism as well as the guidelines on how to avoid plagiarism~\citep{uor_plagiarism}.




% A possible section of you chapter
\section{Critique of the review} % Use this section title or choose a betterone
Describe your main findings and evaluation of the literature. ~\\

% Pleae use this section
\section{Summary} 
Write a summary of this chapter~\\
