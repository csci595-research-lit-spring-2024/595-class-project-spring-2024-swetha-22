\chapter{Literature Review}
\label{ch:lit_rev}

\section{Introduction to Heart Disease}
Cardiovascular diseases pose a significant threat to global health, affecting individuals of all ages. These conditions, including heart disease, heart failure, and irregular heartbeats, have profound physical and emotional impacts on affected individuals. Early detection and effective management are critical in mitigating the adverse effects of heart-related issues and improving patient outcomes.

\section{Background on Machine Learning Models}

\subsection{Logistic Regression}
LR is a statistical method used for binary classification tasks. It models the probability of a binary outcome based on one or more predictor variables. In the context of heart disease prediction, LR can analyze patient parameters such as age, cholesterol levels, and blood pressure to estimate the likelihood of the presence of heart disease. LR is widely used in healthcare research due to its simplicity, interpretability, and ability to handle linear relationships between variables.

\subsection{Naïve Bayes}
NB is a probabilistic classifier based on Bayes' theorem with an assumption of independence between features. Despite its simplistic assumption, NB has been shown to perform well in various classification tasks, including text categorization and medical diagnosis. In heart disease prediction, NB can effectively analyze patient attributes and calculate the conditional probability of heart disease given the observed features.

\subsection{Random Forest}
RF is an ensemble learning method that constructs a multitude of decision trees during training and outputs the mode of the classes (classification) or mean prediction (regression) of the individual trees. RF is known for its robustness and ability to handle high-dimensional data. In heart disease prediction, RF can analyze a large number of patient parameters and identify complex patterns associated with cardiovascular conditions.


\section{Performance Measures for Evaluation}
To evaluate the performance of our machine learning models, we will employ several performance measures, including accuracy, precision, recall, and F1-score. These metrics provide insights into the models' ability to correctly classify instances of heart disease and non-heart disease cases. By evaluating multiple performance measures, we can assess the overall effectiveness of our predictive models and identify areas for improvement.

\section{Description of the Dataset}
Our project utilizes the heart disease dataset sourced from UC Irvine, compiled by \cite{janosi-1988}. This dataset contains various patient attributes, such as age, sex, cholesterol levels, and resting blood pressure, along with the presence or absence of heart disease. We preprocess the dataset to handle missing values and normalize the features to ensure optimal model performance.

\section{Summary of Literature Reviewed}
The literature review highlights the significance of early detection and effective management in combating cardiovascular diseases. Previous studies have demonstrated the utility of machine learning algorithms in predicting heart disease, with research highlighting the importance of feature selection, parameter tuning, and model evaluation. By building upon existing literature and leveraging advanced predictive tools, our project aims to contribute to the ongoing efforts in cardiovascular health and improve patient outcomes.