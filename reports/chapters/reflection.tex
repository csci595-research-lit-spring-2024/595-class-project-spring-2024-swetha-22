\chapter{Reflection}

Undertaking this project has been a significant learning experience for me, extending far beyond the gaining of technical skills. While I did gain proficiency in using various programming languages and tools like LaTeX, the most valuable takeaway from this project was the development of problem-solving skills and research methodology. Through the process of identifying and solving a complex problem such as predicting heart disease, I learned the importance of thorough research inquiry and strategic planning.Figuring out how to predict heart disease was challenging, especially dealing with the complicated dataset and getting the data ready for analysis. Even though I faced some tough moments, like spending a lot of time cleaning the data, I learned a lot in the process. If I were to approach a similar problem in the future, I would focus more on data collection and pre-processing to streamline the model training process. Reflecting on the initial aims and objectives of the project, I realized the need for greater flexibility and adaptability in project planning. While my initial goals were clear and well-defined, the process of research and experimentation led to new insights and adjustments in approach. Overall, this project has not only enhanced my technical skills but also deepened my understanding of the research process and its implications for future work.
