\chapter{Discussion and Analysis}
\label{ch:evaluation}

The Discussion and Analysis chapter evaluates and interprets the results obtained from our heart disease prediction project. We analyze the performance of the machine learning models, including Logistic Regression, Random Forest, and Naive Bayes, in predicting heart disease based on patient data.

\section{Significance of the findings}
The findings of this project hold significant implications for the field of cardiovascular health. By successfully training and evaluating machine learning models for heart disease prediction, we have demonstrated the potential of these models as valuable tools for early detection and risk assessment. The high accuracy and balanced precision-recall trade-off achieved by the Logistic Regression model, in particular, highlight its effectiveness in identifying patients with heart disease. These findings enhance our understanding of the potential applications of machine learning in healthcare and contribute to ongoing efforts to improve patient outcomes in cardiovascular diseases.

\section{Limitations} % please discuss limitation of the project 
But, there are some things we need to think about. We only used one dataset, which might not cover all types of patients. Also, how good our models are might depend on how good the data is and what we look at. We need more research to check if our findings work in different hospitals and with different patients. Plus, we need to make sure our models make sense for doctors to use and don't give them wrong information.

\section{Summary}
In summary, the Discussion and Analysis chapter provides a comprehensive evaluation of the results obtained from our heart disease prediction project. The findings provide the potential of machine learning models, particularly Logistic Regression, in predicting heart disease and improving patient outcomes. While the results are promising, it is essential to consider the limitations and potential implications for future research and clinical practice. 